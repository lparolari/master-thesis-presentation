\documentclass{article}
\usepackage[utf8]{inputenc}

\title{Talk for my MS thesis}
\author{luca.parolari23@gmail.com}
\date{December 16, 2021}

\begin{document}

\maketitle

\section{Talk}

Buongiorno, io sono Luca Parolari e oggi presento il lavoro che ho
svolto durante il tirocinio per la mia tesi, il cui titolo è ``Weakly
Supervised VT grounding based on Concept Similarity''. Durante la
presentazione discuterò il problema del vt grounding motivandone la
rilevanza. Mostrerò poi alcuni degli approcci più rilevanti fatti dai
ricercatori in quest'ambito per poi passare alla nostra soluzione,
discutendone i risultati e tirando poi le somme sul lavoro svolto.

Il visual textual grounding è definito come l'attività di
localizzazione di porzioni di un'immagine referenziate da una frase.
Il problema infatti utilizza due input: l'immagine, che nell'esempio è
il cane che gioca con la palla, e una frase. Nell'esempio, un sistema
di phrase grounding deve essere in grado di localizzare i tre oggetti
(cane, palla e prato) data la frase in questione. Seppur questo possa
sembrare un problema banale per gli umani, per le macchine non lo è
affatto perché richiede la comprensione semantica di entrambe le
modalità visive e testuali e la collaborazione delle due modalità per
produrre le localizzazioni. Il phrase grounding è un problema
perticolare perchè richiede diverse annotazioni: in particolare,
dobbiamo identificare le frasi nominali nella frase, distinguere gli
oggetti nell'immagine e dobbiamo includere un'informazione che
specifichi quali sono le localizzazioni corrette (coppie frase
nominale-immagine).

% Se pensiamo ad un bambino che impara a fare
% queste associazioni, data una frase, dovremo insegnarli a distinguere
% i vari concetti nella frase (nell'esempio: il cane, la palla e il
% prato), dovremo insegnarli che il cane, la palla e il prato hanno
% certe caratterstiche sull'immagine, e in ultimo dovrmeo insegnargli
% come localizzare un concetto sull'immagine (ovvero indicare). 

Il phrase grounding è un problema rilevante in computer vision perchè
è il task tramite cui è possibile esprimere problemi più complessi,
come il visual question answering (ovvero il task di rispondere a
domande riguardanti immagini), image retrieval (ovvero trovare
l'immagine descritta testualmente tra un set di immagini) e la
navigazione robotica. Dallo sviluppo del dataset Flickr30k Entities
fino ad oggi sono stati fatti notevoli progressi nella risoluzione del
problema ma, come dicevo prima, per via delle due modalità visiva e
testuale, il task rimane complicato e ambiguo per una macchina.
Inoltre, per via dell'enorme quantità di annotazioni che richiede (se
ripensiamo all'immagine del cane, abbiamo bisogno delle TODO) è
difficile scalare la soluzione in contesti più ampi e migliorare lo
sviluppo tramite la raccolta di altri dataset con più esempi per via
dell'enorme costo sia in termini di tempo che di denaro. Per questo
motivo il problema è stato studiato con diversi tipi di supervisione,
ovvero con diverse tipologie

% In letteratura ci sono molti lavori che studiano questo problema, ma
% molti evidenziano un limite dato dal numero di annotazioni necessario
% per apprenderlo: ad esempio, una tale quantità di annotazioni è un
% forte limite alla raccolta di nuovi dataset più grandi, specialmente
% per il costo in termini di denaro e soldi che 

In letterature sono stati sperimentate diverse strategie per
affrontare il problema del phrase grounding. In ambito weakly
superivsed però, non avendo a disposizione la ground truth (ovver
l'informazione su quale sia la proposal corretta da collegare ad una
frase), non è possibile operare in maniera tradizionale. Per questo
motivo, alcuni lavori tentano di sfruttare la struttura
morfosintattica delle frasi in linguaggio naturale per costruire dei
vincoli che la localizzazione degli oggetti nell'immagine deve
rispettare. Altri invece riformulano il problema del phrase grounding
come un task di image retrieval, ovvero data una frase bisogna cercare
l'immagine descritta dalla frase tra un set di immagini. L'ipotesi è
che imparare a cercare un'immagine data una frase che la descriva
implichi imparare a localizzare porzioni di immagine, ovvero fare
grounding. In altri casi invece si sfrutta la supervisione weak per
imparare a ricostruire le feature originali di immagine e testo da un
codice. In questo caso si utilizza l'approccio encoder-decoder. In
alcuni lavori recenti si nota un maggiore utilizzo di informazioni
aggiuntive come la conoscenza dell'object detector. In particolare,
spesso e volentieri l'object detector è allenato per classificare
categorie di bounding box su un set di etichette, che teoricamente
dovrebbero il contenuto semantico della proposal.


\end{document}
