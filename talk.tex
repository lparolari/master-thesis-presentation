\documentclass{article}
\usepackage[utf8]{inputenc}

\title{Talk for my MS thesis}
\author{luca.parolari23@gmail.com}
\date{December 16, 2021}

\begin{document}

\maketitle

\section{Talk}

\textbf{Stato dell'arte}. In letterature sono stati sperimentate
diverse strategie per affrontare il problema del phrase grounding. In
ambito weakly superivsed però, non avendo a disposizione la ground
truth (ovver l'informazione su quale sia la proposal corretta da
collegare ad una frase), non è possibile operare in maniera
tradizionale. Per questo motivo, alcuni lavori tentano di sfruttare la
struttura morfosintattica delle frasi in linguaggio naturale per
costruire dei vincoli che la localizzazione degli oggetti
nell'immagine deve rispettare. Altri invece riformulano il problema
del phrase grounding come un task di image retrieval, ovvero data una
frase bisogna cercare l'immagine descritta dalla frase tra un set di
immagini. L'ipotesi è che imparare a cercare un'immagine data una
frase che la descriva implichi imparare a localizzare porzioni di
immagine, ovvero fare grounding. In altri casi invece si sfrutta la supervisione weak per imparare a ricostruire le feature originali di immagine e testo da un codice. In questo caso si utilizza l'approccio encoder-decoder. In alcuni lavori recenti si nota un maggiore utilizzo di informazioni aggiuntive come la conoscenza dell'object detector. In particolare, spesso e volentieri l'object detector è allenato per classificare categorie di bounding box su un set di etichette, che teoricamente dovrebbero il contenuto semantico della proposal.

\textbf{La nostra soluzione}.


\end{document}
